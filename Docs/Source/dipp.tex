% program VLNA -- prida ~ tam kde je potřeba je soucasti texlive baliku
\pdfoutput=1
\documentclass[oneside,12pt]{article}
\usepackage[utf8x]{inputenc}
\usepackage[czech]{babel}
\usepackage{dipp}
\usepackage{listings}
\usepackage{pgfplots, pgfplotstable}
\usepackage{filecontents}
\usepackage{comment}
\usepackage{listings}

%
%
% START
\begin{document}

\def\,{\penalty10000\hskip.25em}
\pagestyle{headings}
\cislovat{2}
\bakalarska
\titul{Hra založená na rozšířené realitě pro platformu iOS}{Aleš Kocur}{Ing. David Procházka, Ph.D.}{Brno~2014}

\obsah

\newpage
%
% Přehled  literatury
%
\kapitola{Rešerše}
% 
% Úvod
Myšlenku rozšířené reality (angl. Augmented reality) nastínil již před více než sto lety americký spisovatel Lyman Frank Baum \cite{baum}, ale až v posledních letech s nástupem mobilních technologií získává na svém potencionálu více než kdy předtím. Mobilní technologie a přenositelná zařízení nám umožňují pomocí senzorů snímat realitu a obohacovat ji o uměle vytvořené prvky. Toho se dá využít v celé škále odvětví jako je zábavní průmysl, medicína, marketing, vzdělávání, navigace, sport a mnoho dalšího. Tato práce se bude zabývat aplikací rozšířené reality v zábavním průmyslu, konkrétně hrami. Cíl této práce je vytvořit hru pro mobilní operační systém iOS, která bude obohacena o prvky rozšířené reality.


\sekce{iOS}
Aplikace pro operační systém iOS byly vyvíjeny primárně v jazyku Objective-C, sekundárně C++, až do září roku 2014, kdy byl uveden nový programovací jazyk Swift. Ten se má podle strategie firmy Apple Inc. stát hlavním vývojovým jazykem pro tuto platformu. Práce bude zahrnovat zdrojový kód ve všech třech zmíněných jazycích (framework Metaio, o kterém se zmiňuji níže, je psán v C++) a jako hlavní literární zdroj budu využívat oficiální dokumentace iOS firmy Apple nazvaná \textit{iOS developer library} \cite{apple_library}. V práci se budu opírat zejména o část dokumentace týkající se frameworku \textit{CoreData} \cite{core_data}, která popisuje principy práce s daty v iOS a jejich perzistenci, a déle taky o principy programování pro iOS popsané v knize \textit{Programming iOS6} \cite{neuburg}, která nabízí obecnější pohled na programování pro iOS.

\sekce{Rozšířená realita na iOS}
Existuje řada frameworků poskytující možnosti práce s rozšířenou realitou, ale pouze málo z nich je komplexních a dobře zdokumentovaných, aby mohli sloužit k vytvoření hry. Byly hodnoceny tyto frameworky:
\begin{itemize}
\item 
ARToolKit
\item
Metaio
\item
Qualcomm Vuforia
\item
Augmented kit
\end{itemize}

\podsekce{ARToolKit}
Tento framework poskytuje všechny základní funkce potřebné pro vytváření rozšířené reality, avšak má velmi omezenou dokumentaci a implementaci na iOS není plně kompatibilní s nejnověšími verzemi.

\podsekce{Metaio}
Metaio pro iOS je dostupné jako C++ framework, samotná práce tedy kombinuje programování ve 3 různých jazycích. V dokumentaci k tomuto frameworku lze nalézt příklady základní implementace jednotlivých funkcí. Další literatura k tomuto konkrétnímu produktu neexistuje, nicméně pro bakalářskou práci naprosto vystačí. Dokumentace uvádí měření ukazucí udržení výkonu vykreslování 60 snímků za vteřinu při 200 000 polygonech na iPhone 5S.

\podsekce{Qualcomm Vuforia}
Jedná se o poměrně nový framework (vyvíjen od roku 2011) a nabízí velmi komplexní funkce, které uspokojí i náročnější požadavky. Tento framework disponuje bohatou dokumentací avšak velká škála funkcí se promítá do licencování tohoto frameworku, které je omezeno počtem rozpoznaných objektů u neplacencýh licencí.

\podsekce{Augmented kit}
Framework psaný čistě pro platformu iOS a tedy v Objective-C. Vyniká jednoduchým API a dobrou dokumentací. Framework je vyvíjen teprve od roku 2012 a to se projevuje malou vývojářskou základnou.

\sekce{Metodika}
Prvním a zároveň velmi důležitým krokem bude zvolení správného frameworku na základě kriterií, mezi něž patří rychlost vykreslování, licencování a dokumentace. Poté je potřeba důkladně navrhnout herní systém a jak a do jaké míry se bude hra odehrávat v rozšířené realitě. Dalším krokem bude zaopatření grafických podkladů a modelů pro hru a poslední krok spočívá v samotné implementaci.

\begin{literatura}

\citace{apple_library}{APPLE, 2014}
{
	\autor{APPLE, INC.}
	\nazev{iOS developer library}
	[online]. [cit. 2014-12-18]. Dostupné z: https://developer.apple.com/library/ios/navigation/	
}

\citace{core_data}{APPLE, 2014}{
	\autor{APPLE, INC.}
	\nazev{Core Data Programming Guide}
	[online]. [cit. 2014-12-18]. Dostupné z: https://developer.apple.com/library/ios/documentation/Cocoa/Conceptual/ CoreData/cdProgrammingGuide.html
}

\citace{baum}{BAUM, 1901} {
	\autor{L. Frank Baum}
	\nazev{The Master Key: An Electrical Fairy Tale, Founded Upon the Mysteries of Electricity and the Optimism of Its Devotees}
	BiblioBazaar, 2006, ISBN 978-1426409240
}

\citace{neuburg}{NEUBURG, 2013}{
	\autor{NEUBURG, Matt}
	\nazev{Programming iOS 6. 3rd edition}
	Sebastopol: O´Reilly, 2013, xxvii, 1154 s. ISBN 978-1-449-36576-9.
}

\citace{swift}{APPLE, 2014}{
	\autor{APPLE, INC.}
	\nazev{The Swift Programming Language}
	[online]. [cit. 2014-12-18]. Dostupné z: https://itunes.apple.com/us/book-series/swift-programming-series/id888896989?mt=11
}

\citace{c++}{STROUSTRUP, 2013}{
	\autor{STROUSTRUP, Bjarne}
	\nazev{The C++ Programming Language}
	 Addison Wesley; 4 edition. 2013, 1368 s. ISBN 978-0321563842.
}

\citace{bimber}{BIMBER, 2005}{
	\autor{BIMBER, Oliver}
	\nazev{Spatial augmented reality: merging real and virtual worlds}
	 Wellesley: A K Peters, 2005, xiii, 369 s. ISBN 15-688-1230-2
}

\citace{furht}{FURHT, 2011}{
	\autor{FURHT, B.}
	\nazev{Handbook of augmented reality}
	New York, NY: Springer, 2011. 746 s. ISBN 978-1-4614-0063-9.
}

\citace{metaio}{Metaio, 2014}{
	\autor{Metaio, gmbh}
	\nazev{Metaio SDK Documentation}
	[online]. [cit. 2014-12-18]. Dostupné z: http://dev.metaio.com/sdk/documentation/
}

\citace{metaio_polycount}{Metaio, 2015}{
	\autor{Metaio, gmbh}
	\nazev{Metaio SDK Documentation - General guidelines}
	[online]. [cit. 2015-02-16]. Dostupné z: http://dev.metaio.com/sdk/documentation/content-creation/3d-animation/polygon-count/general-guidelines/
}

\end{literatura}

\end{document}

% program VLNA -- prida ~ tam kde je potřeba je soucasti texlive baliku

\pdfoutput=1
\documentclass[twoside,12pt]{article}
% \usepackage[utf8]{inputenc} 
\usepackage[utf8x]{inputenc}
\usepackage[czech]{babel}
\usepackage{dipp}
\usepackage{listings}
\usepackage{pgfplots, pgfplotstable}
\usepackage{filecontents}
\usepackage{comment}
\usepackage{listings}

%
%
% START
\begin{document}

\def\,{\penalty10000\hskip.25em}
\pagestyle{headings}
\cislovat{2}

\bakalarska

\titul{Hra založená na rozšířené realitě pro platformu iOS}{Aleš Kocur}{Ing. David Procházka, Ph.D.}{Brno~2015}


\podekovani{
Rád bych poděkoval.. 
}

\prohlasenimuz{V~Brně dne \today}
 
\newpage\null\thispagestyle{empty}\newpage

\abstrakt{
Tato práce se zabývá rozšířenou realitou a jejím využitím ve hrách na~platformě iOS. Zkoumá aktuální přístupy her, které rozšířenou realitu využívají. Práce se zabývá jejich koncepty interakce a přidaných hodnot oproti hrám normálním. Dále jsou představeny frameworky nabízející práci s rozšířenou realitou na mobilní platformě iOS a jejich srovnání na základě vhodnosti pro tvorbu hry. Poslední částí práce je návrh vlastní hry a následná implementace s použitím nejlépe ohodnoceného frameworku.
}{}

\abstract{
EN abstract
}{}



\obsah

% abstrakt
% úvodní text
% Aktuální hry a jejich koncepty
% závěr
% seznam pouzité literatury
% přílohy, rejstříky a seznamy



%
% Přehled použité literatury
%
%\kapitola{Přehled použité literatury}


%
% Úvod a cíl práce
%
\kapitola{Úvod a~cíl práce}

% 
% Úvod
\sekce{Úvod}



% 
% Cíl práce
\sekce{Cíl práce}
Cílem práce je prozkoumat možnosti her v rozšířené realitě na mobilní plaformě iOS a vytvořit hru, které bude rozšířené reality využívat jako obohacujícího prvku pro hráče. Prvním a zároveň velmi důležitým krokem je zvolení správného frameworku na základě výše uvedených kritérií. Poté je potřeba důkladně navrhnout herní systém a jak a do jaké míry se bude hra odehrávat v rozšířené realitě. Dalším krokem bude zaopatření grafických podkladů a modelů pro hru a poslední krok spočívá v samotné implementaci.

%
% Aktuální hry a jejich koncepty
%
\kapitola{Aktuální hry a jejich koncepty}

% 
% ARHrrrr!
\sekce{ARHrrrr!}


% 
% Nova stranka
\newpage 

%
% Závěr
%
\kapitola{Závěr}
Závěr

%
% Návrhy na vylepšení
\sekce{Návrhy na vylepšení}

%
%
% Přílohy
\kapitola{Přílohy}
Na přiloženém CD jsou k dispozici tyto přílohy: 
\begin{itemize}
\item priloha1
\end{itemize}


% 
% Literatura
% 
\begin{literatura}

\end{literatura}


\end{document}
